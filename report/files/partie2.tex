\section{La phase de développement}

\subsection{Les attentes atteintes}

L'ensemble du cahier des charges minimal a été implémenté. De plus, la difficulté est proportionnelle au niveau du joueur. En effet, le nombre d'ennemis dépend du niveau du joueur, et donc des ennemis qu'il a déjà vaincu. Nous avons aussi ajouté une attaque à usage limitée. Des items sont laissés par les ennemis vaincus pour régénérer les points de vie du héros ou son sort limité. Enfin, la vue a été adaptée pour que la carte est redimensionnable en cours de partie. 

\subsection{Ce qui reste inachevé}

Malheureusement, nous n'avons pas réussi à mettre en pratique nos autres idées. Cependant, nous avons tout de même préparé l'implémentation d'autres fonctionnalités. Ainsi, même si l'éditeur de cartes n'a pas été réalisé, le code a été adapté afin de pouvoir charger des fichiers textes formatés pour faciliter son ajout par la suite. 

De la même manière, bien que la vue soit prête pour la modification des touches, et qu'il existe une fonction permettant de modifier les contrôles utilisateurs, nous n'avons pas eu le temps de lier les deux.

\subsection{Les problèmes rencontrés}

	\subsubsection{Quelques problèmes de communication}
 
En nous confrontant pour la première fois à un travail de groupe de cette ampleur, nous avons rencontré quelques difficultés, dont certaines communicationnelles. En effet, lors de la recherche de sprites, nous avons commencé par les boules de feu ; puis nous nous sommes concertés pour la recherche des boules de glace. Cependant à cause d'un quiproquo, nous avons temporairement fini avec des sprites de cornets de glace. 

	\subsubsection{De la mémoire et des fuites}

Après plusieurs longues séances de codage consécutives

	\subsubsection{Gestion événementielle}
	
	Nous n'arrivions pas dans un premier temps à gérer l'appuie de plusieurs touches en même temps. Nous avons alors changer notre Wait\_Event par un Poll\_Event. Ce dernier a comme avantage de garder en mémoire plusieurs évènements, contrairement à Wait\_Event qui les gèrent un par un. Les évènements étant stockés dès le premier appel de la fonction Poll\_Event, un switch(Poll\_Event) n'est pas nécessaire. Nous avons ensuite décidé de créer un tableau de booléen contenant les états de chaque touche nécessaire au jeu. Il a donc fallu contrôler à la fois lorsqu'une touche est pressée, mais aussi lorsqu'elle ne l'est plus. Ce tableau nous a permis de gérer les évènements de manière plus lisible, mais aussi de gérer grâce à un if l'état de plusieurs touches à la fois, permettant de gérer l'appuie de plusieurs touches. 

	\subsubsection{La vue et le modèle}
Collisions et vue déplaçable.
