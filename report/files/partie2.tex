\section{La phase de développement}

Comme pour tout projet qui se respecte, nous avons placé la plupart de nos objectifs trop hauts et nous n'avons pas été capable de les atteindre. Le problème étant que notre planification n'a pris en compte que nos heures de cours, et pas les partiels, les autres projets et les participations associatives.

\subsection{Les attentes atteintes}

L'ensemble du cahier des charges minimal a été implémenté. De plus, la difficulté est proportionnelle au niveau du joueur. En effet, dépendent de cela la vie des ennemis ainsi que leur attaque. 
Nous avons aussi ajouté une attaque à usage limitée dont les munitions peuvent être récoltées en détruisant des ennemis ou des bâtiments.

\subsection{Ce qui reste inachevé}

Malheureusement, nous n'avons pas réussi à mettre en pratique nos autres idées. Cependant, il est à noter que même si l'éditeur de cartes n'a pas été fait, le code a été adapté afin de pouvoir charger des fichiers textes formatés pour faciliter son ajout par la suite. Il en va de même pour la sélection d'une carte particulière.  

\subsection{Quelques problèmes de communication}
 
En nous confrontant pour la première fois à un travail de groupe de cette ampleur, nous avons rencontré quelques difficultés, dont certains communicationnelles. En effet, lors de la recherche de sprites libres de droit, nous avons commencé par les boules de feu ; puis nous nous sommes concertés pour la recherche des boules de glace. Cependant à cause d'un quiproquo, nous avons fini avec des sprites de cornets de glace. 
