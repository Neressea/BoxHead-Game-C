\section{La phase de développement}

\subsection{Les attentes atteintes}

L'ensemble du cahier des charges minimal a été implémenté. De plus, la difficulté est proportionnelle au niveau du joueur. En effet, le nombre d'ennemis dépend du niveau du joueur, et donc des ennemis qu'il a déjà vaincu. Nous avons aussi ajouté une attaque à usage limitée. Des items sont laissés par les ennemis vaincus pour régénérer les points de vie du héros ou son sort limité. Enfin, la vue a été adaptée pour que la carte est redimensionnable en cours de partie. 

\subsection{Ce qui reste inachevé}

Malheureusement, nous n'avons pas réussi à mettre en pratique nos autres idées. Cependant, nous avons tout de même préparé l'implémentation d'autres fonctionnalités. Ainsi, même si l'éditeur de cartes n'a pas été réalisé, le code a été adapté afin de pouvoir charger des fichiers textes formatés pour faciliter son ajout par la suite. 

De la même manière, bien que la vue soit prête pour la modification des touches, et qu'il existe une fonction permettant de modifier les contrôles utilisateurs, nous n'avons pas eu le temps de lier les deux.

\subsection{Les problèmes rencontrés}

	\subsubsection{Quelques problèmes de communication}
 
En nous confrontant pour la première fois à un travail de groupe de cette ampleur, nous avons rencontré quelques difficultés, dont certaines communicationnelles. En effet, lors de la recherche de sprites, nous avons commencé par les boules de feu ; puis nous nous sommes concertés pour la recherche des boules de glace. Cependant à cause d'un quiproquo, nous avons temporairement fini avec des sprites de cornets de glace. 

	\subsubsection{De la mémoire et des fuites}

Après plusieurs longues séances de codage consécutives

	\subsubsection{}

	\subsubsection{}
