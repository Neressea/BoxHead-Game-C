\section{Présentation du sujet} % Pas de numérotation
\addcontentsline{toc}{section}{Présentation du sujet} % Ajout dans la table des matières

\subsection{Un rapide résumé}
Le jeu que nous nous sommes proposés de réaliser est un beat’em’all et tower defense s’inspirant du jeu BoxHead. Le
principe général est de survivre sur une carte à plusieurs vagues successives d’ennemis. Pour se faire,
le joueur incarne un personnage en vue à la troisième personne qui peut se déplacer, attaquer à distance avec des sorts 
et construire des défenses.

\begin{figure}
    \includegraphics[width=0.5\textwidth]{./images/snapshot1.png}
    \includegraphics[width=0.5\textwidth]{./images/snapshot1.png}
    \caption{Snapshots}
\end{figure}

\subsection{Une licence pour les protéger tous}

Le logiciel et l'ensemble des sources sont sous licence CC BY-NC-SA 3.0 FR. \\
Le partage et l'adaptation du logiciel sont permises à condition de ne pas l'utiliser
commercialement, d'indiquer les changements effectués, de référencer l’œuvre originale et 
de conserver la même licence lors d'une diffusion ultérieure.

\subsection{Le cahier des charges}

Afin de gérer au mieux le projet, nous nous proposons d’établir un cahier des charges minimal dont les
caractéristiques se retrouveront dans toute version finale du projet. En plus de cela, nous établirons une liste
non ordonnée de fonctionnalités qu’il semble de prime abord intéressant d’implémenter. Leur mise en place
découlera de leur difficulté ainsi que du temps disponible.
Cahier des charges minimal :
\begin{itemize}
\item Affichage graphique de la partie (joueur, carte et ennemis)
\item Lancement d’une partie sur une carte unique
\item Interagir avec le clavier pour se déplacer dans l’environnement
\item Gestion de l’interaction avec l’environnement (collision, ...)
\item Utilisation d’une arme pour se défendre
\item Création d’un dispositif de défense
\item Implémentation de l’IA des ennemis (repérer, poursuivre le joueur et éviter les obstacles)
\item Implémentation de l’IA générale de la partie qui gérera la création d’ennemis au cours du temps
\item Menu d’accueil
\end{itemize}
Fonctionnalités optionnelles :
\begin{itemize}
\item Mode coopératif à deux en réseau ou sur le même clavier
\item Mode deathmatch entre deux joueurs
\item Création/achat de nouvelles armes
\item Création de systèmes défensifs supplémentaires
\item Mini mode aventure
\item Difficulté croissante des ennemis dans le temps (nouveaux ennemis plus rapide, plus résistants, ...)
\item Menu pause
\item Choix de la difficulté générale
\item Capacités spéciales à usage limité
\item Ajoute de cartes supplémentaires
\item Éditeur de cartes
\end{itemize}